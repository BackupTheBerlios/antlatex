\section{Entwicklungsgeschichte}
\label{sec:entwicklungsgeschichte}

\begin{description}
% \item[0.0.6\_2 (2006-01-12)]\ \\[-2em]
%   \begin{itemize}
%   \item 
%   \end{itemize}
\item[0.0.7 (2006-02-09)]\ \\[-2em]
  \begin{itemize}
  \item Es k�nnen jetzt mehrere \BibTeX{}-Aufrufe im \LaTeX-Task
    eingetragen werden, mit dem Schalter \code{inloop} kann gesteuert
    werden, ob \BibTeX{} in der \LaTeX-Schleife aufgerufen werden
    soll.
  \end{itemize}
\item[0.0.6\_2 (2006-01-12)]\ \\[-2em]
  \begin{itemize}
  \item \code{path}-Attribut f�r den Pfad zu (PDF)\LaTeX{} darf nicht
    mit getCanonicalPath() behandelt werden, weil unter Unix einige
    Programme ihre Funktionalit�t �ber den Programmnamen erhalten und
    dementsprechend verlinkt sind. (Gefunden von Alphonse Bendt)
  \end{itemize}

\item[0.0.6 (2006-01-09)]\ \\[-2em]
  \begin{itemize}
  \item \code{path}-Attribut f�r den Pfad zu (PDF)\LaTeX{}
    erg�nzt und damit die Semantik von \code{command}
    ge�ndert. (Gefunden von Alphonse Bendt)
  \item Die Argumente \code{output-directory, aux-directory, job-name}
    f�r \LaTeX{} umgesetzt und die Subaufrufe von \BibTeX, makeindex
    und \GlossTeX{} angepasst. Die restlichen Argumente k�nnen jetzt
    mit \code{passThruLaTeXParameters} getrennt mit Semikolons an
    \LaTeX{} weitergegeben werden.
  \item Mit dem Attribut \code{figbib} wird das Package \code{figbib}
    im \LaTeX-Lauf ber�cksichtigt.
  \item Keine \LaTeX-Datei gefunden bzw. angegeben f�hrt zu einer
    Fehlermeldung wenn \code{verbose} auf \code{on} geschaltet ist.
  \end{itemize}
        
\item[0.0.5\_2 (2006-01-06)]\ \\[-2em]
  \begin{itemize}
  \item replace-Aufruf auf Java 1.4 ge�ndert. (Gefunden von Alphonse Bendt)
  \item Keine \LaTeX-Datei gefunden bzw. angegeben f�hrt zu einer vern�nftigen Fehlermeldung.
  \end{itemize}
        
\item[0.0.5 (2005-12-29)]\ \\[-2em]
  \begin{itemize}
  \item Bei den Tasks \code{LaTeX} und \code{LaTeXTask} konnte nur
    eine Datei angegeben werden, jetzt kann auch das Element
    $<$fileset$>$ benutzt werden. (Gefunden von Alphonse Bendt)
  \end{itemize}

\item[0.0.4 (2005-11-26)]\ \\[-2em]
  \begin{itemize}
  \item Es gab zwar die M�glichkeit mit $<$fileset$>$ einen Filter f�r
    $<$makeindex$>$ zu setzen, doch dieses Element wurde innerhalb
    eines $<$latex$>$ Tasks nicht ausgewertet. (Gefunden von Thomas
    Reuter)
  \item Das XML-basierte \textbf{jxGlossar} wurde mit eingebunden.
  \item Das Element $<$delete$>$ wurde ersetzt durch
    $<$deletetempfile$>$
  \end{itemize}

\item[0.0.3 (2005-09-23)]\ \\[-2em]
  \begin{itemize}
  \item Das Attribut \code{if} stimmte nicht mit der normalen Semantik
    von Ant �berein, deswegen wurde jetzt das Attribut \code{run} zum
    Ein/Ausschalten des Tasks eingef�hrt.
  \end{itemize}

\item[0.0.2 (2005-09-07)]\ \\[-2em]
  \begin{itemize}
  \item Beim Task \code{LaTeX} die Elemente \code{makeindex},
    \code{bibtex} und \code{glosstex} zugelassen um die
    \glqq{}komplexeren\grqq{} Attribute durchreichen zuk�nnen.
  \item Mit dem \code{if}-Attribut k�nnen jetzt die Tasks ein- bzw.
    ausgeschaltet werden.
  \end{itemize}

\item[0.0.1 (2005-09-02)]\ \\[-2em]
  \begin{itemize}
  \item Erste Ver�ffentlichung
  \end{itemize}

\end{description}

