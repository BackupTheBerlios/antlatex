\appendix

\section{Beispieldateien}
\label{sec:beispiel:dateien}

\begin{center}
  \textsb{Beispiel einer Build-Datei}
\ifHtml
\Link[beispiele/build.xml]{}{}build.xml\EndLink
\fi
\end{center}
%{\footnotesize
\verbatimfile{beispiele/build.xml}
%}

\begin{center}
  \textsb{Beispiel einer Definionsdatei}  
\ifHtml
\Link[beispiele/build.properties]{}{}build.properties\EndLink
\fi
\end{center}
%{\footnotesize
\verbatimfile{beispiele/build.properties}
%}

\begin{center}
  \textsb{Grundlegende Definionen der Tasks und Targets}  
\ifHtml
\Link[beispiele/latex.xml]{}{}latex.xml\EndLink
\fi
\end{center}
%{\footnotesize
\verbatimfile{beispiele/latex.xml}
%}
%\begin{latexonly}
% Ausdruck des Glossarverzeichnisses
%\printglossary
%\end{latexonly}
% \begin{htmlonly}
%   \newglossary
%   \input{hauptdokument.glx}
% \end{htmlonly}
% Ausdruck des Literaturverzeichnisses. Es muessen evtl. die Datenbanken noch
% angepasst werden.
%\bibliographystyle{geralpha}
%\bibliography{tex_database,programming_database,datafiles/local}

%\listoffigures
%\listoftables
% % Diese Verzeichnisse werden in der HTML-Version nicht /vernuenftig/ ausgedruckt.
% %\begin{latexonly}
% \listofstruktos %% Spezialbefehl bei Verwendung von strukto.sty
% %bzw. giaAPI.sty zur Ausgabe eines Verzeichnis der Struktogramme.
% \listofsources %% Spezialbefehl bei der Verwendung von giaAPI.sty zur Ausgabe
% %eines Verzeichnis der Listings.
% %\end{latexonly}

% Zum Abschluss das Stichwortverzeichnis.
% Vorsicht: Es wird an dieser auf zweispaltig umgeschaltet und bedeutet ein
% Seitenumbruch zur rechten Seite, d.h. es kann eine Leerseite auftauchen.
%\printindex
